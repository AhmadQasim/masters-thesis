\chapter{\abstractname}\label{abstract}
For deep learning image classification, typically a large amount of annotated data is needed. For real world images, it is usually cheap and easy to get annotations but for biomedical images, annotations require time from experts, which is expensive and in short supply. Therefore, it is crucial to use methods which fully exploit the annotated data available and minimize the need of annotations. Active learning algorithms help with identifying more informative data, which mitigates the need of a large number of annotations for achieving good performance. However, most of active learning algorithms are tested on real world images and it remains to be found that how do they perform on biomedical images, which have high class imbalance, low levels of contrast between different colors and a high level of similarity between different classes. Moreover, active learning algorithms usually do not exploit the unlabeled data available, in addition to labeled data. \\
In this thesis, different strategies for combining active learning with pre-training and semi-supervised learning for achieving good performance on the task of biomedical image classification, are investigated. First, three active learning algorithms, three pre-training methods, and two training strategies are tested on a dataset which consists of almost 20,000 white blood cell images, classified into ten different classes. It is found that the performance of active learning algorithms can be boosted by using pre-training with self-supervised learning and pre-training with pre-trained ImageNet weights. The performance can be further improved by using semi-supervised learning. A specific combination of different active learning algorithms, pre-training methods and training strategies is found by carrying out an extensive grid search on three biomedical image datasets. An improvement of 3\% to 14\% macro recall was observed over conventional annotation-efficient classification strategies by using the recommended combination found through the grid search. It is expected that this recommended strategy will show an improvement in results whenever the problem of scarce-annotation is faced.


